\documentclass[
12pt,
ngerman
]{scrreprt}

\usepackage[T1]{fontenc}
\usepackage[utf8]{inputenc}
\usepackage{babel}


\begin{document}
\author{Maximilian Hauser}
\title{Allgemeine Schlüsselqualifikation \\
\LaTeX - praktische Anwendung in wissenschaftlichen Arbeiten}

\maketitle
\tableofcontents

\chapter{Kurstag 1 - Kapitel 1 \& 2}
Diese Section beinhaltet einen fiktiven Text, welcher der Darstellung eines Fließtextes gilt um zu zeigen, dass ein automatischer Zeilenumbruch gemacht wird.
\section{Aufzählungsliste}
\begin{itemize}
  \item Ich bin der erste Aufzählungspunkte
  \item Ich bin der zweite Aufzählungspunkte
  \item Ich bin der dritte Aufzählungspunkte
\end{itemize}
\section{nummerierte Liste}
\begin{enumerate}
  \item Ich bin der erste nummerierte Punkt
  \item Ich bin der zweite nummerierte Punkt
  \item Ich bin der dritte nummerierte Punkt
\end{enumerate}
\section{Besondere Zeichen}
\subsection{deutsche Anführungszeichen}
Deutsche Anführungszeichen werden mit folgenden Befehlen erzeugt: \\
\begin{itemize}
  \item "` wird mit \verb!"`! oder \verb!\glqq! erzeugt.
  \item "' wird mit \verb!"'! oder \verb!\grqq! erzeugt.
\end{itemize}
Beispiel: \\
"`Ich bin ein Satz in deutschen Anführungszeichen"' \\
\glqq Ich bin ein Satz in deutschen Anführungszeichen\grqq
\subsection{französische Anführungszeichen}
Französische Anführungszeichen werden mit folgenden Befehlen erzeugt: \\
\begin{itemize}
  \item "< wird mit \verb!"<! oder \verb!\flqq! erzeugt.
  \item "> wird mit \verb!">! oder \verb!\frqq! erzeugt.
\end{itemize}
Beispiel: \\
"<Ich bin ein Satz in französischen Anführungszeichen"> \\
\flqq Ich bin ein Satz in französischen Anführungszeichen\frqq
\subsection{englische Anführungszeichen}
Englische Anführungszeichen werden mit folgenden Befehlen erzeugt: \\
\begin{itemize}
  \item `` wird mit \verb!``! erzeugt.
  \item '' wird mit \verb!''! erzeugt.
\end{itemize}
Beispiel: \\
``Ich bin ein Satz in englischen Anführungszeichen''
\subsection{Sonderzeichen}
\subsubsection{\%}
Das Sonderzeichen \% kann mit dem Befehl \verb!\%! im Text erzeugt werden.
\subsubsection{\&}
Das Sonderzeichen \& kann mit dem Befehl \verb!\&! im Text erzeugt werden.
\subsubsection{\#}
Das Sonderzeichen \# kann mit dem Befehl \verb!\#! im Text erzeugt werden.
\subsubsection{\$}
Das Sonderzeichen \$ kann mit dem Befehl \verb!\$! im Text erzeugt werden.
\subsubsection{\_}
Das Sonderzeichen \_ kann mit dem Befehl \verb!\_! im Text erzeugt werden.
\subsubsection{\{}
Das Sonderzeichen \{ kann mit dem Befehl \verb!\{! im Text erzeugt werden.
\subsubsection{\}}
Das Sonderzeichen \} kann mit dem Befehl \verb!\}! im Text erzeugt werden.
\subsection{Zeilenumbruch}
Ein Zeilenumbruch kann mit dem Befehl \verb!\\! im Text erzeugt werden.
\subsection{Textgruppen}
Textgruppen können mit \verb!{!Text\verb!}! erstellt werden. Diese können dann mit Befehlen wie \verb!\emph! gestaltet werden.
\end{document}
