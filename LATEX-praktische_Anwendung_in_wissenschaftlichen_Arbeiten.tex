\documentclass[
12pt,
ngerman
]{scrreprt}

\usepackage[T1]{fontenc}
\usepackage[utf8]{inputenc}
\usepackage{babel}

\usepackage{enumitem}
\usepackage{multicol}

\usepackage{varioref}
\usepackage{hyperref}
\usepackage{cleveref}

\usepackage{chngcntr}
\counterwithout{footnote}{chapter}

\usepackage{footmisc}

\usepackage{scrlayer-scrpage}
\pagestyle{scrheadings}
\clearpairofpagestyles

\automark[chapter]{chapter}
\renewcommand*{\chaptermarkformat}{\chapapp~\thechapter:\enskip}
\chead{\headmark}
\cfoot{\pagemark}
\renewcommand*{\chapterpagestyle}{scrheadings}


\begin{document}
\author{Maximilian Hauser}
\title{Allgemeine Schlüsselqualifikation \\
\LaTeX \:- praktische Anwendung in wissenschaftlichen Arbeiten}

\tableofcontents

\chapter{Kurstag 1 - Kapitel 1 \& 2}
\label{cha:chap1-2}
\label{blatt:1}
Diese Section beinhaltet einen fiktiven Text, welcher der Darstellung eines Fließtextes gilt um zu zeigen, dass ein automatischer Zeilenumbruch gemacht wird.

\section{Aufzählungsliste}
\label{sec:item-list}
\begin{itemize}
  \item Ich bin der erste Aufzählungspunkte
  \item Ich bin der zweite Aufzählungspunkte
  \item Ich bin der dritte Aufzählungspunkte
\end{itemize}

\section{nummerierte Liste}
\label{sec:enum-list}
\begin{enumerate}
  \item Ich bin der erste nummerierte Punkt
  \item Ich bin der zweite nummerierte Punkt
  \item Ich bin der dritte nummerierte Punkt
\end{enumerate}

\section{Besondere Zeichen}
\label{sec:special-cases}

\subsection{deutsche Anführungszeichen}
\label{subsec:ger-qoute}
Deutsche Anführungszeichen werden mit folgenden Befehlen erzeugt: \\
\begin{itemize}
  \item "` wird mit \verb!"`! oder \verb!\glqq! erzeugt.
  \item "' wird mit \verb!"'! oder \verb!\grqq! erzeugt.
\end{itemize}
Beispiel: \\
"`Ich bin ein Satz in deutschen Anführungszeichen"' \\
\glqq Ich bin ein Satz in deutschen Anführungszeichen\grqq

\subsection{französische Anführungszeichen}
\label{subsec:fr-qoute}
Französische Anführungszeichen werden mit folgenden Befehlen erzeugt: \\
\begin{itemize}
  \item "< wird mit \verb!"<! oder \verb!\flqq! erzeugt.
  \item "> wird mit \verb!">! oder \verb!\frqq! erzeugt.
\end{itemize}
Beispiel: \\
"<Ich bin ein Satz in französischen Anführungszeichen"> \\
\flqq Ich bin ein Satz in französischen Anführungszeichen\frqq

\subsection{englische Anführungszeichen}
\label{subsec:gb-qoute}
Englische Anführungszeichen werden mit folgenden Befehlen erzeugt: \\
\begin{itemize}
  \item `` wird mit \verb!``! erzeugt.
  \item '' wird mit \verb!''! erzeugt.
\end{itemize}
Beispiel: \\
``Ich bin ein Satz in englischen Anführungszeichen''

\subsection{Sonderzeichen}
\label{subsec:special-cases}

\subsubsection{\%}
Das Sonderzeichen \% kann mit dem Befehl \verb!\%! im Text erzeugt werden.

\subsubsection{\&}
Das Sonderzeichen \& kann mit dem Befehl \verb!\&! im Text erzeugt werden.

\subsubsection{\#}
Das Sonderzeichen \# kann mit dem Befehl \verb!\#! im Text erzeugt werden.

\subsubsection{\$}
Das Sonderzeichen \$ kann mit dem Befehl \verb!\$! im Text erzeugt werden.

\subsubsection{\_}
Das Sonderzeichen \_ kann mit dem Befehl \verb!\_! im Text erzeugt werden.

\subsubsection{\{}
Das Sonderzeichen \{ kann mit dem Befehl \verb!\{! im Text erzeugt werden.

\subsubsection{\}}
Das Sonderzeichen \} kann mit dem Befehl \verb!\}! im Text erzeugt werden.

\subsection{Zeilenumbruch}
\label{subsec:linebreak}
Ein Zeilenumbruch kann mit dem Befehl \verb!\\! im Text erzeugt werden.

\subsection{Textgruppen}
\label{subsec:textgroups}
Textgruppen können mit \verb!{!Text\verb!}! erstellt werden. Diese können dann mit Befehlen wie \verb!\emph! gestaltet werden.

\section{Verhindern von Zeilenumbrüchen}
\label{sec:no-linebreak}
Um Zeilenumbrüche ziwschen zwei Wörtern oder Ziffern verhindern zu können, gibt es zwei Arten von Leerzeichen welche dafür verwendert werden können:
\begin{itemize}
  \item mit einer \verb!~! kann ein Zeilenumbruch zwischen zwei Wörtern verhindert werden.
  \item mit dem Befehl \verb!\nolinebreak! lässt sich ebenfalls ein Zeilenumbruch verhindern.
\end{itemize}
Sollte man nun vergessen haben, wie man einen Zeilenumbruch erstellt, falls dieser doch gewünscht ist, kann man bei \cref{subsec:linebreak} nochmals nachlesen.

\section{Zeilenabstand}
\label{sec:linespace}
Um den Zeilenabstand auf "`eineinhalbzeilig"' bzw. 1,5 ändern zu können, kann das Paket \verb!setspace! verwendet werden. Dieses wird über \verb!\usepackage{setspace}! eingebunden. Den Zeilenabstand kann man dort ebenfalls für das ganze Dokument definieren indem man es in die [] hineinschreibt. Hier gibt es folgende Optionen:
\begin{itemize}
  \item \verb!\usepackage[doublespace]{setspace}!: Der Zeilenabstand wird auf 2 gesetzt.
  \item \verb!\usepackage[onehalfspace]{setspace}!: Der Zeilenabstand wird auf 1,5 gesetzt.
  \item \verb!\usepackage[singelspace]{setspace}!: Der Zeilenabstand wird auf 1 gesetzt.
\end{itemize}
Um nur für einen Bereich den Zeilenabstand anpassen zu können, kann man mit
\begin{verbatim}
  \begin{setspace}{...}
    ...
  \end{setspace}{...}
\end{verbatim}
den Bereich definieren.

\section{Aufzählungszeichen ändern}
\label{sec:itemize-labels}
Um die Aufzählungszeichen ändern zu können, müssen wir den Befel "`renewen"'. Dies können wir mit \verb!\renewcommand{}! vornehmen. Um das Aufzählungszeichen der ersten Ebene zu ändern geben wir den Befehl \verb!\renewcommand{\labelitemi}{Gewünsches Item}! ein. Bei den weiteren Ebenenen wird die Römische Ziffer am Ende des \verb!\labelitem! Befehls um 1 erhöht, sprich: \verb!\labelitemi!, \verb!\labelitemii!, \verb!\labelitemiii!, \verb!\labelitemiv!\\
Um das gewünsche Ergebnis aus dem Aufgabenblatt zu erzielen müssen wir folgende Befehle vor der Liste ausführen:
\begin{verbatim}
  \renewcommand{\labelitemi}{$\rightarrow$}
  \renewcommand{\labelitemii}{$\rightarrow$}
\end{verbatim}
Somit erzielen wir folgendes Ergebnis:
\begin{itemize}
  \renewcommand{\labelitemi}{$\rightarrow$}
  \renewcommand{\labelitemii}{$\rightarrow$}
  \item Erste Ebene
  \begin{itemize}
    \item Zweite Ebene
  \end{itemize}
\end{itemize}
Wollen wir den Effekt nur innerhalb einer bestimmten Liste erzielen, dann können wir die \verb!\renewcommand!-Befehle nach dem \verb!\begin{itemize}! einfügen.

\section{Nummerierung von arabisch auf römisch umstellen}
\label{sec:enum-label}
Um die Zahlen von einer nummerierten Liste von arabischen Zahlen auf römische Zahlen zu ändern, müssen wir zuerst das Paket\verb!enumitem! laden. Dieses wird über \verb!\usepackage{enumitem}! eingebunden. \\
Daraufhin können wir die nummerierten Listen mit Argumenten in [eckigen Klammern] erweitern. In unserem Fall brauchen wir hierfür das Argument \verb![label=\roman*]! bzw. \verb![label=\Roman*]! für große Ziffern. Dadruch bekommen wir folgende Ergebnisse:
\begin{multicols}{2}
  \begin{enumerate}[label=\roman*]
    \item Erster Punkt
    \item Zweiter Punkt
    \item Dritter Punkt
  \end{enumerate}
  \emph{Liste mit kleinen römischen Ziffern}
  \begin{enumerate}[label=\Roman*]
    \item Erster Punkt
    \item Zweiter Punkt
    \item Dritter Punkt
  \end{enumerate}
  \emph{Liste mit großen römischen Ziffern}
\end{multicols}
Sollten wir nun doch anstatt einen nummerierten Liste eine unnummerierte Liste mit eigenen Aufzählungszeichen wollen, lohnt es sich bei \cref{sec:itemize-labels} vorbei zu schauen.

\chapter{Kurstag 1 - Kapitel 3 \& 4}
\label{cha:chap3-4}
\label{blatt:2}

\section{Nummerierung der Fußnoten über alle Kapitel}
\label{sec:footnotes}
Eine Fußnote\footnote{Eine Fußnote ist eine Anmerkung, die im Druck-Layout aus dem Fließtext ausgelagert wird, um den Text flüssig lesbar zu gestalten. Eine Fußnote kann als „Anmerkung, Legende, Bemerkung, Quellenangabe, Übersetzung oder weiterführende Erklärung zu einem Wort oder einer Textpassage“ dienen.} kann man mit dem Befehl \verb!\footnote{name}! erstellen. Dass diese nun auch in einer PDF auf die Fußnote verlinkt werden, binden wir die Pakete \verb!varioref!\footnote{Einbinden über \textbackslash usepackage\{varioref\}}, \verb!hyperref!\footnote{Einbinden über \textbackslash usepackage\{hyperref\}} und \verb!cleverref!\footnote{Einbinden über \textbackslash usepackage\{cleverref\}} ein. Dass die Fußnotennummerierung nun nicht in jedem Kapitel bei 1 wieder beginnt, binden wird das Paket \verb!chngcntr!\footnote{Einbinden über \textbackslash usepackage\{chngcntr\}} ein und nutzen den Befehl \verb!\counterwithout{footnote}{chapter}!. Somit wird der Counter der Fußnoten nicht bei jedem Kapitel wieder auf 1 zurückgesetzt.
\section{Ergänzung des \LaTeX \:Dokuments}
Die Aufgaben a)\footnote{mehrseitiger Text}, b)\footnote{zwei Kapitel}, c)\footnote{weitere Abschnitte und Unterabschnitte} und e)\footnote{beliebige Texte mit ordentlicher Länge} wurden in diesem ganzen Dokument bereits bearbeitet. Um nun  die Anfangsthemen überfliegen zu können, kann man diese im \cref{cha:chap1-2} nochmals nachlesen. Da wir nun die Aufgabe d)\footnote{an mehreren Stellen Querverweise auf verschiedene Teile Ihres Textes} ebenfalls erledigt haben, können wir uns nun Aufgabe f)\footnote{Testen Sie die scrlayer-scrpage-Variante für lebende Kolumnentitel in Ihrem
Dokument der Klasse scrreprt} widmen. Dafür binden wir das Paket \verb!scrlayer-scrpage!\footnote{\label{fn:scrlayer-scrpage}Einbinden über \textbackslash usepackage\{scrlayer-scrpage\}} ein. Wir wählen auch gleich noch den Seitenstil scrheadings\footnote{\label{fn:scrheadings}Auswahl über \textbackslash pagestyle\{scrheadings\}} aus, wodurch LaTeX mitgeteilt wird, dass Kopf- und Fußzeilen mit dem \verb!scrlayer-scrpage!\footref{fn:scrlayer-scrpage} Paket gesetzt werden sollen. Der Befehl \verb!\clearpairofpagestyles! leert alle sechs Platzhalter, sodass wir diese nach unseren Wünschen befüllen können. Wenn wir nun \verb!\chead{\headmark}! mit den weiteren Befehlen \verb!\automark[chapter]{chapter}! und \\ \verb!\renewcommand*{\chaptermarkformat}{\chapapp~\thechapter:\enskip}! verwenden, wird der Name des aktuellen Kapitels auf den Unterseiten des Kapitels oben in der Mitte angezeigt.
Um die Seitenzahlen im Fußbereich anzeigen zu können, verwenden wir \verb!\cfoot{\pagemark}!. Dass nun alles auch auf den ersten Seiten des Kapitels angezeigt wird, verwenden wir den Befehl \verb!\renewcommand*{\chapterpagestyle}{scrheadings}! um den Style der Chapter Seiten auch auf den Seitensil \verb!scrheadings!\footref{fn:scrheadings} stellen.
\end{document}
