Praxis zu Tabellen und Formeln

In Tabelle *** hier Verweis auf die untenstehende Tabelle einfügen *** ist eine über eine Seite hinausreichende Tabelle zu sehen. 

% hier die Tabelle einfügen und sie nah wie möglich am Ziel-PDF gestalten

Durchschnittsnoten: Für und Wider

Schulnotendurchschnitten sollte man skeptisch gegenüberstehen, insbesondere da sie ja als arithmetischer Mittelwert, also als Summe aller Werte geteilt durch ihre Anzahl errechnet werden:

% hier erste Formel einfügen; achten Sie auf die Ausrichtung der beiden Zeilen und die Art der Nummerierung

Warum?
% Setzen Sie in den foldenden Absätzen an den betreffenden Stellen die korrekten Anführungszeichen:
Schulnoten sollen die Schulleistung messen. Wir alle wissen, eine Eins ist besser als eine  Zwei; diese ist besser als eine  Drei etc.
Doch um WIEVIEL ist die eine Note besser resp. schlechter als eine andere? Es sind zunächst nur Aussagen auf dem Niveau einer Rangskala (besser, schlechter, ...). Die Zahlen könnten auch  Sieben,  Dreizehn,  Siebzehn etc. lauten ...
Die Verwendung des arithmischen Mittelwerts setzt aber ein metrisches Skalenniveau voraus, bei dem die Zahlen einen konkreten definierten Wert auf der Skala haben. Erst dann macht es Sinn, sie zu addieren und durch ihre Anzahl zu teilen.

Richtigerweise sollte man auf dem Rangskalenniveau den Median als Maß für die  zentrale Tendenz verwenden. Der Median ist definiert als mittlerer Wert in der nach Größe geordneten Reihe aller Messwerte. Bei einer ungeraden Anzahl von Messwerten gibt es diesen Wert in der Mitte direkt, bei einer geraden Anzahl von Messwerten gibt es zwei Werte in der Mitte, die dann, so die Definition des Medians, arithmetisch gemittelt werden: beide Werte werden addiert und durch zwei geteilt: 

% hier die zweite Formel einfügen, siehe Kursbegleitendes Skript

So schön es ist, jetzt zu wissen, wie es richtig geht - mit derartigen Fragestellungen sind Kultusminister und auch viele Lehrer wohl doch überfordert.